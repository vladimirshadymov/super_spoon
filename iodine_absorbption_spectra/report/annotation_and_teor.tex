\begin{center}
	\vspace{0.5cm}{\parbox{16cm}{\small{\centering{\textbf{Аннотация}\\
					\hspace{0.6cm} В этом отчёте изложены результаты выполнения лабораторной работы «Изучение электронно-колебательных спектров поглощения двухатомных молекул на примере молекулы I$_2$». Приводится	краткая теория молекулярных спектров, рассматриваются особенности спектра поглощения молекулярного йода в диапазоне температур $T = 300\div350$ К, приводится описание экспериментальной установки и методики регистрации спектров при различных температурах. 
					Исследуется электронно-колебательно-вращательный спектр поглощения паров иода. Спектр поглощения, исследуемый в работе, соответствует электронному переходу $^1\Sigma^+_g$ и лежит в области длин волн $490 \leq \lambda \leq 650$ нм.
				}}}}
\end{center}

\textbf{\emph{Цель работы:}} изучение структуры электронно-колебательно спектра поглощения двухатомных молекул, определение из спектроскопических данных основных молекулярных постоянных с использованием статистических методов обработки результатов эксперимента на ЭВМ.
\section{Теоретическое введение}
При соединении атомов в молекулы их электронные оболочки объединяются.
Электромагнитные поля, возникающие в процессе образования молекулы при
сближении электронных оболочек атомов, уже не являются сферически
симметричными, как в атоме. Появление новых степеней свободы движения
частиц молекулы отражается на структуре её энергетических уровней и,
следовательно, на структуре молекулярного спектра. Наличие в молекуле двух
и более положительно заряженных ядер существенно усложняет рассмотрение
поведения системы заряженных частиц. Если в атоме с помощью квантовой
механики рассматривается распределение вероятности нахождения электронов
в поле только одного ядра, то в случае молекулы необходимо рассматривать
как распределение вероятности нахождения электронов в поле двух и более
ядер, так и вероятность нахождения ядер в пространстве относительно
заданной системы координат.

Из всех свойств атомов и молекул наиболее важно знание их внутренней
энергии $E$. Фундаментальным уравнением, связывающим энергию системы с её
волновой функцией $\Psi$, является стационарное уравнение Шредингера:
\begin{equation}
\label{shr}
\hat H\Psi=E\Psi,
\end{equation}
где $\hat H$ --- оператор полной энергии (гамильтониан). Для того, чтобы
теоретически определить возможные стационарные энергетические состояния
системы частиц (атома, молекулы, иона) и затем по ним рассчитать спектры
или термодинамические функции, необходимо составить для системы оператор
Гамильтона $\hat H$ и решить уравнение \eqref{shr}. Однако точно в аналитическом виде
уравнение Шредингера решается только для простейших модельных систем,
например, гармонического осциллятора, жёсткого ротатора и некоторых
других. Для молекулы уравнение Шредингера настолько усложняется, что его
точное аналитическое решение возможно только для простейшей двухатомной
молекулы – иона при фиксированном положении ядер. Однако бурное
развитие вычислительной техники и создание всё более совершенных и
мощных ЭВМ в последнее время дало новый импульс квантовомеханическим
расчётам молекулярных состояний. В дальнейшем изложении мы будем
ограничиваться в основном наиболее простым случаем двухатомной молекулы,
для которой теория молекулярных спектров наиболее полно разработана.

Для большинства практических задач молекулярной спектроскопии достаточно
точным является приближённое представление полной волновой функции
молекулы в виде произведения
\begin{equation}
\label{psi_var}
\Psi = \Psi_e\Psi_v\Psi_r,
\end{equation}
где индексы $e, v, r$ относятся соответственно к движению электронов,
колебательному движению ядер и к вращательному движению молекулы как
целого. Это позволяет решать уравнение Шредингера \eqref{shr} отдельно для
электронной, колебательной и вращательной волновых функций. В
приближении \eqref{psi_var} полную внутреннюю энергию молекулы можно представить в виде суммы
\begin{equation}
\label{energy}E = E_e+ E_v+E_r,
\end{equation}
где $E_e$ - энергия электронной оболочки молекулы, $E_v$ - энергия колебаний ядер молекулы, $E_r$ - энергия вращения молекулы. Разделение полной волновой функции по типу \eqref{psi_var}, а следовательно, и разделению энергии молекулы на сумму энергий отдельных видов движения возможно при
условии
\begin{equation}
E_e \gg E_v \gg E_r
\end{equation}
что в большинстве случаев выполняется.